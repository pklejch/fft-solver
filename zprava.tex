\documentclass[12pt]{article}
\usepackage{epsf,epic,eepic,eepicemu}
%\documentstyle[epsf,epic,eepic,eepicemu]{article}
\usepackage[czech]{babel}
\usepackage[utf8]{inputenc} % LaTeX source encoded as UTF-8
\usepackage{graphicx}
\usepackage{listings}
\usepackage{subcaption}
\usepackage{hyperref}

\begin{document}
%\oddsidemargin=-5mm \evensidemargin=-5mm \marginparwidth=.08in
%\marginparsep=.01in \marginparpush=5pt \topmargin=-15mm
%\headheight=12pt \headsep=25pt \footheight=12pt \footskip=30pt
%\textheight=25cm \textwidth=17cm \columnsep=2mm \columnseprule=1pt
%\parindent=15pt\parskip=2pt

\begin{center}
\bf Semestrální projekt MI-PAP 2015/2016:\\[5mm]
    Paralelní algoritmus pro FFT\\[5mm]
       Petr Klejch\\
\today
\end{center}

\section{Definice problému a popis sekvenčního algoritmu}

%%o cem
%
%Definici problému
%Popis sekvenčního algoritmu a jeho implementace
\subsection{Obecný popis úlohy}
\subsubsection{Diskrétní Fourierova transformace}
Diskrétní Fourierova transformace (DFT) je transformace převádějící vstupní signál na signál vyjádřený pomocí funkcí $\sin$ a $\cos$. Obecně se diskrétní Fourierova transformace vstupního vektoru o velikosti $N$ spočítá jako:
$$X_{k} = \sum_{n=0}^{N-1} x_n \cdot e^{-2 \pi i  k n / N},$$ což odpovídá složitosti $\mathcal{O}(N^2)$. 
Byl však ale objeven algoritmus, který spočítá diskrétní Fourierovu transformaci v čase $\mathcal{O}(N\log{N})$ jménem rychlá Fourierova transformace (FFT).
\subsubsection{Rychlá Fourierova transformace}
Rychlá Fourierova transformace je algoritmus typu rozděl a panuj, který rekurzivně dělí vstupní vektor na menší částí, na tyto části opět rekurzivně aplikuje algoritmus FFT a poté zkombinuje výsledné části. Nejčastější implementace je varianta, která dělí vstupní vektor na poloviny a tudíž vstupní vektor musí být velikosti $N^2$. 

\subsection{Popis sekvenčního algoritmu}
Jelikož je dle zadání vstupní vektor o velikosti $N=2^{k_{1}}*3^{k_{2}}$, nemohla být použita pouze nejčastější implementace FFT - radix-2 FFT, která dělí vektor rekurzivně na poloviny.\par
Bylo využito obecnější varianty Cooley-Tukey algoritmu FFT, který předpo\-kládá vstupní vektor o velikosti $N=Q*P$. Tento algoritmus převede vstupní vektor do matice o rozměrech $Q*P$, aplikuje $Q\times$ algoritmus DFT na vektor o velikosti $P$ (řádky matice), poté vynásobí všechny prvky matice tzv. twiddle faktory ($e^{-2 \pi i  k / N}$), následně se matice transponuje a aplikuje se $P\times$ algoritmus DFT na vektor o velikosti $Q$ (opět řádky matice).\par
Jelikož víme, že vstupní vektor je velikosti $N=2^{k_{1}}*3^{k_{2}}$, a tedy $Q=2^{k_{1}}$ a $P=3^{k_{2}}$, je možné použít na řádky matice radix-2, resp. radix-3 FFT variantu, která dělí vektor na poloviny, resp. na třetiny.\par
Sekvenční algoritmus se tedy skládá z těchto kroků:
\begin{enumerate}
\item Faktorizace $N$ na čísla $Q$ a $P$, kde $Q=2^{k_{1}}$ a $P=3^{k_{2}}$.
\item Převedení vstupního vektoru do matice.
\item Aplikace $Q\times$ algoritmu radix-3 FFT na řádky matice.
\item Vynásobením všech prvků matice twiddle faktory.
\item Transpozice matice.
\item Aplikace $P\times$ algoritmu radix-2 FFT na řádky matice.
\item Přečtení matice po řádcích, která obsahuje výsledný transformovaný vektor. 
\end{enumerate} 

\section{Popis paralelního algoritmu a jeho implementace v OpenMP}
Sekvenční algoritmus byl vhodný pro paralelní zpracování, a tak vyžadoval minimum úprav. Paralelizovány byly zejména cykly, které počítají $Q\times$ (resp. $P\times$) FFT po řádcích. Jelikož se každý řádek zpracovává zvlášť, nebyla nutná žádná synchronizace (krok (3) a (6)). Dále byla paralelizována část, která transponuje matici (krok (5)) a část která násobí každý prvek matice twiddle faktory (krok (4)).
\subsection{Vektorizace}
Na architektuře x86 drtivá část cyklů nebyla vektorizována, kvůli datovým závislostem. Dále některé cykly obsahují podmínky nebo volání funkcí, které zabraňují vektorizaci.\par
Na architektuře Xeon Phi byla situace o něco lepší. Kompilátor byl schopen vektorizovat několik cyklů. Bohužel cykly, které jsou prováděny nejčastěji vektorizovány nebyly, proto zrychlení oproti nevektorizované verzi je v nejlepším případě přibližně 6 \%, jak lze vidět z tabulek \ref{tab:mereni2} a \ref{tab:mereni3}. \par
Schopnost vektorizace by se dala zlepšit přepsáním kódu, který by byl v souladu s pravidly pro vektorizaci.
\newpage
\subsection{Naměřené výsledky a grafy}
\subsubsection{Architektura x86}
\begin{table}[h!]
\centering
\begin{tabular}{ | c || c | c | c | }
  \hline
  & \multicolumn{3}{ |c| }{Velikost instance} \\
  \hline
  \hline
  Počet vláken & N=4 478 976 ($2^{11}*3^{7}$) & N=5 038 848 ($2^{8}*3^{9}$) & N=6 718 464 ($2^{10}*3^{8}$)\\
   \hline
   1 & 7,1 & 7,9 & 10,6  \\
  \hline
   2 & 3,7 & 4,2  & 5,6 \\
  \hline
   4 & 2,0 & 2,1 & 2,9 \\
  \hline
   6 & 1,3 & 1,4 & 2  \\
  \hline
   8 & 1,0 & 1,2 & 1,6  \\
  \hline
   12 & 1,0 & 1,0 & 1,4  \\
  \hline
   24 & 0,6 & 0,8 & 0,8  \\
  \hline
  
\end{tabular}
\caption{Délka běhu úlohy v sekundách v závislosti na velikosti vstupních dat a počtu vláken.}
\label{tab:mereni}
\end{table}

\begin{figure}[!h]
	\centering
    \includegraphics[width=\textwidth]{figures/graf.eps}
    \caption{Graf délky běhu úlohy v závislosti na počtu vláken.}
    \label{fig:graph2}
\end{figure}

\begin{figure}[!h]
	\centering
    \includegraphics[width=\textwidth]{figures/speedup_x86.eps}
    \caption{Graf zrychlení v závislosti na počtu vláken.}
    \label{fig:speedup_x86}
\end{figure}
\newpage



\subsubsection{Architektura Xeon Phi}

%bez Vektorizace
\textbf{Vypnutá vektorizace}
\begin{table}[!h]
\centering
\begin{tabular}{ | c || c | c | c | }
  \hline
  & \multicolumn{3}{ |c| }{Velikost instance} \\
  \hline   
  \hline
  Počet vláken & N=4 478 976 ($2^{11}*3^{7}$) & N=5 038 848 ($2^{8}*3^{9}$) & N=6 718 464 ($2^{10}*3^{8}$)\\
   \hline
   1 & 25,9 & 31,6 & 41,9  \\
  \hline
   61 & 3,4 & 5,7 & 5,3 \\
  \hline
   122 & 2,8 & 3,9 & 4 \\
  \hline
   244 & 2,1 & 2,4 & 3,4  \\
  \hline
  
\end{tabular}
\caption{Délka běhu úlohy s vypnutou vektorizací v sekundách v závislosti na velikosti vstupních dat a počtu vláken.}
\label{tab:mereni2}
\end{table}
\begin{figure}[!h]
	\centering
    \includegraphics[width=\textwidth]{figures/xeon_novect.eps}
    \caption{Graf délky běhu úlohy v závislosti na počtu vláken na architektuře Xeon Phi s vypnutou vektorizací.}
    \label{fig:xeon_novect}
\end{figure}
\newpage


%Vektorizace
\textbf{Zapnutá vektorizace}
\begin{table}[!h]
\centering
\begin{tabular}{ | c || c | c | c | }
  \hline
  & \multicolumn{3}{ |c| }{Velikost instance} \\
  \hline
  \hline
  Počet vláken & N=4 478 976 ($2^{11}*3^{7}$) & N=5 038 848 ($2^{8}*3^{9}$) & N=6 718 464 ($2^{10}*3^{8}$)\\
   \hline
   1 & 24,9 & 30 & 40,3  \\
  \hline
   61 & 3,3 & 5,5 & 5 \\
  \hline
   122 & 2,6 & 3,7 & 4,3 \\
  \hline
   244 & 2,2 & 2,2 & 3,3  \\
  \hline
  
\end{tabular}
\caption{Délka běhu úlohy se zapnutou vektorizací v sekundách v závislosti na velikosti vstupních dat a počtu vláken.}
\label{tab:mereni3}
\end{table}

\begin{figure}[!h]
	\centering
    \includegraphics[width=\textwidth]{figures/xeon_vect.eps}
    \caption{Graf délky běhu úlohy v závislosti na počtu vláken na architektuře Xeon Phi se zapnutou vektorizací}
    \label{fig:xeon_vect}
\end{figure}

\begin{figure}[!h]
	\centering
    \includegraphics[width=\textwidth]{figures/speedup_xeon.eps}
    \caption{Graf zrychlení v závislosti na počtu vláken na architektuře Xeon Phi se zapnutou vektorizací.}
    \label{fig:speedup_xeon}
\end{figure}

\clearpage
\subsection{Zhodnocení}
Na architektuře x86 paralelní algoritmus relativně dobře škáloval a v určitých instancích bylo dosaženo lineárního zrychlení, jak lze vidět na grafu \ref{fig:speedup_x86}. \par
Bohužel na architektuře Xeon Phi algoritmus škáloval velmi špatně, jak lze vidět na grafu \ref{fig:speedup_xeon}. Bylo to pravděpodobně způsobeno špatnou granularitou úlohy a nevyužitou vektorizací. Jelikož vláknům byla přidělována velká část práce, a jelikož výpočetní vlákna na architektuře Xeon Phi jsou výkonnostně slabší než na architektuře x86, práce nebyla efektivně rozvržena, což vedlo ke špatným časovým výsledkům. 

\section{Popis paralelního algoritmu a jeho implementace v CUDA}
\subsection{Úpravy v algoritmu}
Původní algoritmus byl částečně upraven. Mezi nejvýraznější změny patřilo změna reprezentace vektoru v paměti. \par 
V původním algoritmu byla využita standardní třída complex, která reprezentovala komplexní číslo. V původním programu se pracovalo s polem těchto tříd. Jelikož však operace mezi komplexními čísly (sčítání a násobení) jsou přetížené operátory $+$ a $*$, funkce které zajišťují tyto operace nebylo možné volat z kernelu. Program byl proto přepsán a v paměti byl vektor komplexních čísel reprezentován dvěma poli datového typu float. Jedno pole na \textit{i}-tém prvku obsahovalo reálnou složku komplexního čísla, druhé pole na \textit{i}-tém prvku obsahovalo imaginární složku. V průběhu implementace však byla objevena knihovna thrust, která obsahuje implementaci třídy complex, mající operátory, které se dají volat z kernelu (mají identifikátor device). Rozdělení vektoru do dvou polí datového typu float bylo nakonec zanecháno, ale pro samotné operace mezi komplexními čísly bylo využito knihovny thrust.\par
Dále musela být funkcí cudaDeviceSetLimit navýšena velikost zásobníku i haldy. Jelikož je algoritmus FFT rekurzivní pro větší instance docházelo k přetečení zásobníku. Z důvodu dopředné alokace paměti pomocných bufferů bylo také nutné navýšit velikost haldy.
\par
Mezi další úpravy patřila dopředná alokace pomocných bufferů, aby se běh algoritmu nezpomaloval voláním funkcí malloc a free za běhu.
\subsection{Počet vláken a velikost bloku}
Jak je napsáno výše, celý algoritmus se po dá rozdělit do čtyř hlavních částí. A tedy právě tyto části se staly kernely.
\begin{enumerate}
	\item Běh radix3-FFT po řádcích matice.
	\item Transpozice matice.
	\item Přenásobení matice tzv. twiddle faktory.
	\item Běh radix2-FFT po řádcích matice. 
\end{enumerate}
Jelikož je vstupní vektor velikosti $N$ umístěn do matice o velikosti $Q$ a $P$, radix3-FFT je puštěn v $Q$ vláknech, transpozice matice a přenásobení matice je puštěno v $N$ vláknech a radix2-FFT je puštěno v $P$ vláknech. Muselo být tedy vždy vypočteno s kolika bloky o dané velikosti bude kernel spuštěn. Počet bloků se počítal pomocí toho výrazu   $$\lceil počet\_potřebných\_vláken / počet\_vláken\_v\_bloku \rceil.$$
Tedy např. pro instanci $N=6718464$ a bloku o velikosti 512 vláken bylo pro běh radix3-FFT použito 13 bloků, na běh transpozice a přenásobení 13122 bloků a na radix2-FFT 2 bloky.
\subsection{Mez na přepnutí DFT algoritmu}
FFT algoritmus se dle zadání neměl provádět až do pole velikosti 1, ale od nějaké meze se měl přepnout na triviální DFT algoritmus. Při testování velikosti meze však bylo zjištěno, že se zvětšující se mezí se výpočetní čas prodlužuje a pro malou mez byla úspora času zanedbatelná, na hranici chyby měření.\par
Mez, která dosahovala nejlepších výsledků byla 4 či 8. Úspora času pravdě\-podobně nebyla tak veliká, jelikož algoritmus DFT nelze provádět in-place. Při výpočtu DFT je i pro poslední prvek výstupního vektoru nutná znalost prvního prvku vstupního vektoru. Tedy výstupní vektor musí být v samostatném poli. Po výpočtu se musí výstupní vektor překopírovat na místo vstupního, což byla pravděpodobně příčina zpomalení.
\newpage
\subsection{Naměřené výsledky a grafy na architektuře CUDA}
\begin{table}[!hb]
\centering
\begin{tabular}{ | c || c | c | c | }
  \hline
  & \multicolumn{3}{ |c| }{Velikost instance} \\
  \hline   
  \hline
  Velikost bloku & N=4 478 976 ($2^{11}*3^{7}$) & N=5 038 848 ($2^{8}*3^{9}$) & N=6 718 464 ($2^{10}*3^{8}$)\\
   \hline
   32 & 2,45 & 17,28 & 5,66  \\
  \hline
   64 & 2,45 & 17,39 & 5,64 \\
  \hline
   128 & 2,48 & 17,39 & 5,6 \\
  \hline
   256 & 2,52 & 18,52 & 6  \\
  \hline
   512 & 2,68 & 18,42 & 6,26  \\
  \hline
   1024 & 2,93 & 18,44 & 6,69  \\
  \hline
  
\end{tabular}
\caption{Délka běhu úlohy v sekundách v závislosti na velikosti vstupních dat a velikosti bloku.}
\label{tab:mereni_cuda}
\end{table}
\begin{figure}[!hb]
	\centering
    \includegraphics[width=\textwidth]{figures/cuda.eps}
    \caption{Graf délky běhu úlohy v závislosti na velikosti bloku na architektuře CUDA.}
    \label{fig:cuda}
\end{figure}

\newpage
\subsection{Zhodnocení algoritmu na architektuře CUDA}
Jak je vidět v grafu \ref{fig:cuda} velikost bloku měla na délku běhu algoritmu vliv. Nejlepší výsledky byly dosaženy s velikostí bloku 32. Rozdíl v času mezi velikostí bloku 32 a 1024 v případě největší instance byla jedna sekunda.\par
Z grafu si lze také všimnou, že instance N=5 038 848 běžela podstatně déle než ostatní. To bylo způsobeno faktem, že matice byla velmi obdélníková ($256*19683$), radix3-FFT byl počítán pouze pomocí 256 vláken a tak nebyla plně využita paralelizace.
\section{Závěr}
Úkolem semestrální práce bylo naimplementovat paralelní FFT algoritmus se vstupním vektorem o velikosti $N=2^{k_{1}}*3^{k_{2}}$. Algoritmus se podařilo naimplementovat na všechny zadané architektury - x86, Xeon Phi a CUDA.\par
Co se týče rychlosti, zvítězila s přehledem architektura x86. To bylo pravděpodobně zapříčeno faktem, že tato architektura má výpočetně nejsilnější vlákna a ta převážila nad paralelizací. Tento algoritmus nebyl efektivně paralelizován, zejména co se týče granularity. V případě obdélníkových matic je tento fakt nejvíce znatelný, kdy některá vlákna mají přiděleno příliš mnoho práce. Na tuto skutečnost nejvíce doplatila architektura CUDA, která obsahuje velmi velké množství výpočetně slabých vláken.\par
Při srovnání architektur CUDA a Xeon Phi při využití maxima vláken jsou časy běhů pro některé instance srovnatelné. I zde, ale CUDA ztrácí v případě obdélníkové vstupní matice. 
\appendix
\end{document}