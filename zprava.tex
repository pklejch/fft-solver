\documentclass[12pt]{article}
\usepackage{epsf,epic,eepic,eepicemu}
%\documentstyle[epsf,epic,eepic,eepicemu]{article}
\usepackage[czech]{babel}
\usepackage[utf8]{inputenc} % LaTeX source encoded as UTF-8
\usepackage{graphicx}
\usepackage{listings}
\usepackage{subcaption}
\usepackage{hyperref}

\begin{document}
%\oddsidemargin=-5mm \evensidemargin=-5mm \marginparwidth=.08in
%\marginparsep=.01in \marginparpush=5pt \topmargin=-15mm
%\headheight=12pt \headsep=25pt \footheight=12pt \footskip=30pt
%\textheight=25cm \textwidth=17cm \columnsep=2mm \columnseprule=1pt
%\parindent=15pt\parskip=2pt

\begin{center}
\bf Semestrální projekt MI-PAP 2015/2016:\\[5mm]
    Paralelní algoritmus pro FFT\\[5mm]
       Petr Klejch\\
\today
\end{center}

\section{Definice problému a popis sekvenčního algoritmu}

%%o cem
%
%Definici problému
%Popis sekvenčního algoritmu a jeho implementace
\subsection{Obecný popis úlohy}
\subsubsection{Diskrétní Fourierova transformace}
Diskrétní Fourierova transformace (DFT) je transformace převádějící vstupní signál na signál vyjádřený pomocí funkcí $\sin$ a $\cos$. Obecně se diskrétní Fourierova transformace vstupního vektoru o velikosti $N$ spočítá jako:
$$X_{k} = \sum_{n=0}^{N-1} x_n \cdot e^{-2 \pi i  k n / N},$$ což odpovídá složitosti $\mathcal{O}(N^2)$. 
Byl však ale objeven algoritmus, který spočítá diskrétní Fourierovu transformaci v čase $\mathcal{O}(N\log{N})$ jménem rychlá Fourierova transformace (FFT).
\subsubsection{Rychlá Fourierova transformace}
Rychlá Fourierova transformace je algoritmus typu rozděl a panuj, který rekurzivně dělí vstupní vektor na menší částí, na tyto části opět rekurzivně aplikuje algoritmus FFT a poté zkombinuje výsledné části. Nejčastější implementace je varianta, která dělí vstupní vektor na poloviny a tudíž vstupní vektor musí být velikosti $N^2$. 

\subsection{Popis sekvenčního algoritmu}
Jelikož je dle zadání vstupní vektor o velikosti $N=2^{k_{1}}*3^{k_{2}}$, nemohla být použita pouze nejčastější implementace FFT - radix-2 FFT, která dělí vektor rekurzivně na poloviny.\par
Bylo využito obecnější varianty Cooley-Tukey algoritmu FFT, který předpo\-kládá vstupní vektor o velikosti $N=Q*P$. Tento algoritmus převede vstupní vektor do matice o rozměrech $Q*P$, aplikuje $Q\times$ algoritmus DFT na vektor o velikosti $P$ (řádky matice), poté vynásobí všechny prvky matice tzv. twiddle faktory ($e^{-2 \pi i  k / N}$), následně se matice transponuje a aplikuje se $P\times$ algoritmus DFT na vektor o velikosti $Q$ (opět řádky matice).\par
Jelikož víme, že vstupní vektor je velikosti $N=2^{k_{1}}*3^{k_{2}}$, a tedy $Q=2^{k_{1}}$ a $P=3^{k_{2}}$, je možné použít na řádky matice radix-2, resp. radix-3 FFT variantu, která dělí vektor na poloviny, resp. na třetiny.\par
Sekvenční algoritmus se tedy skládá z těchto kroků:
\begin{enumerate}
\item Faktorizace $N$ na čísla $Q$ a $P$, kde $Q=2^{k_{1}}$ a $P=3^{k_{2}}$.
\item Převedení vstupního vektoru do matice.
\item Aplikace $Q\times$ algoritmu radix-3 FFT na řádky matice.
\item Vynásobením všech prvků matice twiddle faktory.
\item Transpozice matice.
\item Aplikace $P\times$ algoritmu radix-2 FFT na řádky matice.
\item Přečtení matice po řádcích, která obsahuje výsledný transformovaný vektor. 
\end{enumerate} 

\section{Popis paralelního algoritmu a jeho implementace v OpenMP}
Sekvenční algoritmus byl vhodný pro paralelní zpracování, a tak vyžadoval minimum úprav. Paralelizovány byly zejména cykly, které počítají $Q\times$ (resp. $P\times$) FFT po řádcích. Jelikož se každý řádek zpracovává zvlášť, nebyla nutná žádná synchronizace (krok (3) a (6)). Dále byla paralelizována část, která transponuje matici (krok (5)) a část která násobí každý prvek matice twiddle faktory (krok (4)).
\subsection{Vektorizace}
Na architektuře x86 drtivá část cyklů nebyla vektorizována, kvůli datovým závislostem. Dále některé cykly obsahují podmínky nebo volání funkcí, které zabraňují vektorizaci.\par
Na architektuře Xeon Phi byla situace o něco lepší. Kompilátor byl schopen vektorizovat několik cyklů. Bohužel cykly, které jsou prováděny nejčastěji vektorizovány nebyly, proto zrychlení oproti nevektorizované verzi je v nejlepším případě přibližně 6 \%, jak lze vidět z tabulek \ref{tab:mereni2} a \ref{tab:mereni3}. \par
Schopnost vektorizace by se dala zlepšit přepsáním kódu, který by byl v souladu s pravidly pro vektorizaci.
\newpage
\subsection{Naměřené výsledky a grafy}
\subsubsection{Architektura x86}
\begin{table}[h!]
\centering
\begin{tabular}{ | c || c | c | c | }
  \hline
  & \multicolumn{3}{ |c| }{Velikost instance} \\
  \hline
  \hline
  Počet vláken & N=4 478 976 ($2^{11}*3^{7}$) & N=5 038 848 ($2^{8}*3^{9}$) & N=6 718 464 ($2^{10}*3^{8}$)\\
   \hline
   1 & 7,1 & 7,9 & 10,6  \\
  \hline
   2 & 3,7 & 4,2  & 5,6 \\
  \hline
   4 & 2,0 & 2,1 & 2,9 \\
  \hline
   6 & 1,3 & 1,4 & 2  \\
  \hline
   8 & 1,0 & 1,2 & 1,6  \\
  \hline
   12 & 1,0 & 1,0 & 1,4  \\
  \hline
   24 & 0,6 & 0,8 & 0,8  \\
  \hline
  
\end{tabular}
\caption{Délka běhu úlohy v sekundách v závislosti na velikosti vstupních dat a počtu vláken.}
\label{tab:mereni}
\end{table}

\begin{figure}[!h]
	\centering
    \includegraphics[width=\textwidth]{figures/graf.eps}
    \caption{Graf délky běhu úlohy v závislosti na počtu vláken.}
    \label{fig:graph2}
\end{figure}

\begin{figure}[!h]
	\centering
    \includegraphics[width=\textwidth]{figures/speedup_x86.eps}
    \caption{Graf zrychlení v závislosti na počtu vláken.}
    \label{fig:speedup_x86}
\end{figure}
\newpage



\subsubsection{Architektura Xeon Phi}

%bez Vektorizace
\textbf{Vypnutá vektorizace}
\begin{table}[!h]
\centering
\begin{tabular}{ | c || c | c | c | }
  \hline
  & \multicolumn{3}{ |c| }{Velikost instance} \\
  \hline   
  \hline
  Počet vláken & N=4 478 976 ($2^{11}*3^{7}$) & N=5 038 848 ($2^{8}*3^{9}$) & N=6 718 464 ($2^{10}*3^{8}$)\\
   \hline
   1 & 25,9 & 31,6 & 41,9  \\
  \hline
   61 & 3,4 & 5,7 & 5,3 \\
  \hline
   122 & 2,8 & 3,9 & 4 \\
  \hline
   244 & 2,1 & 2,4 & 3,4  \\
  \hline
  
\end{tabular}
\caption{Délka běhu úlohy s vypnutou vektorizací v sekundách v závislosti na velikosti vstupních dat a počtu vláken.}
\label{tab:mereni2}
\end{table}
\begin{figure}[!h]
	\centering
    \includegraphics[width=\textwidth]{figures/xeon_novect.eps}
    \caption{Graf délky běhu úlohy v závislosti na počtu vláken na architektuře Xeon Phi s vypnutou vektorizací.}
    \label{fig:xeon_novect}
\end{figure}
\newpage


%Vektorizace
\textbf{Zapnutá vektorizace}
\begin{table}[!h]
\centering
\begin{tabular}{ | c || c | c | c | }
  \hline
  & \multicolumn{3}{ |c| }{Velikost instance} \\
  \hline
  \hline
  Počet vláken & N=4 478 976 ($2^{11}*3^{7}$) & N=5 038 848 ($2^{8}*3^{9}$) & N=6 718 464 ($2^{10}*3^{8}$)\\
   \hline
   1 & 24,9 & 30 & 40,3  \\
  \hline
   61 & 3,3 & 5,5 & 5 \\
  \hline
   122 & 2,6 & 3,7 & 4,3 \\
  \hline
   244 & 2,2 & 2,2 & 3,3  \\
  \hline
  
\end{tabular}
\caption{Délka běhu úlohy se zapnutou vektorizací v sekundách v závislosti na velikosti vstupních dat a počtu vláken.}
\label{tab:mereni3}
\end{table}

\begin{figure}[!h]
	\centering
    \includegraphics[width=\textwidth]{figures/xeon_vect.eps}
    \caption{Graf délky běhu úlohy v závislosti na počtu vláken na architektuře Xeon Phi se zapnutou vektorizací}
    \label{fig:xeon_vect}
\end{figure}

\begin{figure}[!h]
	\centering
    \includegraphics[width=\textwidth]{figures/speedup_xeon.eps}
    \caption{Graf zrychlení v závislosti na počtu vláken na architektuře Xeon Phi se zapnutou vektorizací.}
    \label{fig:speedup_xeon}
\end{figure}

\clearpage
\subsection{Zhodnocení}
Na architektuře x86 paralelní algoritmus relativně dobře škáloval a v určitých instancích bylo dosaženo lineárního zrychlení, jak lze vidět na grafu \ref{fig:speedup_x86}. \par
Bohužel na architektuře Xeon Phi algoritmus škáloval velmi špatně, jak lze vidět na grafu \ref{fig:speedup_xeon}. Bylo to pravděpodobně způsobeno špatnou granularitou úlohy a nevyužitou vektorizací. Jelikož vláknům byla přidělována velká část práce, a jelikož výpočetní vlákna na architektuře Xeon Phi jsou výkonnostně slabší než na architektuře x86, práce nebyla efektivně rozvržena, což vedlo ke špatným časovým výsledkům. 
\appendix

\end{document}